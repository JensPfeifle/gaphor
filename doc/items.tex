\documentclass[draft]{article}
\usepackage[papername = a4paper, margin = 3cm]{geometry}
\usepackage{graphicx}
\usepackage{hyperref}

\title{Gaphor Diagram Item Model}
\author{wrobell@pld-linux.org}

\newcommand{\rmodule}[1]{\texttt{#1}}
\newcommand{\rclass}[1]{\texttt{#1}}
\newcommand{\rattr}[1]{\texttt{#1}}
\newcommand{\rstereotype}[1]{\texttt{<<#1>>}}
\newcommand{\warning}[1]{\textbf{WARNING:~#1}}

%\newenvironment{description}
%               {\list{}{\labelwidth\z@ \itemindent-\leftmargin
%                        \let\makelabel\descriptionlabel}}
%               {\endlist}
%\newcommand*\descriptionlabel[1]{\hspace\labelsep
%                                \normalfont\bfseries #1}

% length class, method and attribute block descriptions
\newlength{\dscwidth}
\setlength{\dscwidth}{\textwidth}
\addtolength{\dscwidth}{-7em}

\newlength{\abdscwidth}
\setlength{\abdscwidth}{\textwidth}
\addtolength{\abdscwidth}{-5em}

\newlength{\adscwidth}
\setlength{\adscwidth}{\textwidth}
\addtolength{\adscwidth}{-9em}

\newcommand{\dlabel}[1]{\par\noindent\makebox[6em][r]{\textbf{#1:}}\hspace{1em}}

\newenvironment{entitydesc}{%
\dlabel{Description}%
\begin{minipage}[t]{\dscwidth}
}{%
\end{minipage}
}

\newcommand{\pre}{\dlabel{Pre}~}
\newcommand{\post}{\dlabel{Post}~}

\newenvironment{slist}{%
    \begin{list}{-}{%
        \setlength{\labelwidth}{3pt}%
        \setlength{\leftmargin}{1em}%
        \setlength{\labelsep}{2pt}%
        \setlength{\rightmargin}{0pt}%
        \setlength{\itemsep}{0pt}%
        \setlength{\parsep}{0pt}%
        \setlength{\topsep}{2pt}%
        \setlength{\partopsep}{0pt}%
    }%
}{%
    \end{list}%\vspace{3pt}%
}

% class
\newenvironment{class}[1]{%
\dlabel{\underline{Class}}\texttt{#1}
}{%
\par
}

% attributes
\newenvironment{attrs}{%
\\[1ex]
\noindent\hspace*{5em}\begin{minipage}[t]{\abdscwidth}
}{%
\end{minipage}\\[-2ex]
}

\newcommand{\iattr}[2]{%
\makebox[3em][r]{\texttt{#1}:}\hspace*{1em}%
\begin{minipage}[t]{\adscwidth}
#2%
\end{minipage}\\
}

% methods
\newenvironment{method}[1]{%
\vspace*{2ex}
\dlabel{\underline{Method}}\texttt{#1}
}{%
\par
}

\begin{document}
\maketitle

\section{Basic Classes of Diagram Items}
\begin{class}{DiagramItem}
\begin{attrs}
\iattr{subject}{reference to UML class instance}
\iattr{popup\_menu}{item popup menu definition, i.e. rename, delete, etc.}
\end{attrs}
\begin{entitydesc}
Basic class for all diagram items.

Subject is one to one association
but can be \texttt{None} as in case of LifetimeItem or
CommentLine items.
It can also visualize more than one UML class (see UML
specification), for example
\begin{slist}
\itemsep0pt
\item DecisionNodeItem items can represent combined decision and merge nodes
\item ForkNodeItem items can represent fork and join nodes 
\end{slist}
\end{entitydesc}
\end{class}

\begin{method}{DiagramItem.set\_subject}
\begin{attrs}
\iattr{subject}{reference to UML class instance}
\end{attrs}
\begin{entitydesc}
assign UML class instance to an item
\end{entitydesc}
\end{method}

\subsection{Items and Diacanvas}
Diacanvas library supports different canvas objects
\begin{itemize}
\item canvas element
\item canvas line
\end{itemize}

Because of differences between them, DiagramItem class cannot be directly
associated with canvas line or canvas element. Connection between diacanvas
classes and items is established by other abstract classes. These are
\begin{description}
\item[LineItem]
    canvas line items like dependencies, association, flows, message, etc.

\item[ElementItem]
    canvas element items with shape like class, component, lifeline,
    comment, activity nodes, etc.
            
\item[FeatureItem]
    canvas element, which is part of an item; for example represents
    attributes and methods of classes 
\end{description}

\section{Named Items}

Named items represents these UML classes, which derive from NamedElement.
All named items are editable, so user can double click on an item and
change or enter name of UML object.

We have to distinguish between two kind of named elements. One is related
to canvas elements and second is related to canvas lines. For example, name
of UML object can be positioned inside or outside canvas element (class vs.
initial node), also name can be centered or on left/right side of canvas
element. In case of canvas line based items name can be near head, tail
or centre of a line and name can be put horizontaly or along line path.

\subsection{Canvas Element Named Items}


\subsection{Canvas Line Named Items}
Basic class for line-like items is LineItem class.

\section{Line Items}
\subsection{Line Handles}
Line has at leas two handles. These handles are reffered as head (first
hanndle) and tail (second handle) of line.

When user puts an item on a diagram, then last handle can be moved to
desired point. The same applies to line items. User puts line on diagram,
head is set in mouse cursor point and tail is moved to desired point.

Head and tail concepts are also used in diacanvas in case of canvas line.
Head of line item is at the same end as head of canvas line. The same
applies for tail.

\subsection{Relationships}
See also relationships.txt.

Relationships are implemented using Python descriptors. See Relationship,
DependencyRelationship and AssociationRelationship classes for examples.

\section{Stereotypes}
See also stereotypes.txt.

\section{Align of Item Elements}

\begin{figure}
\begin{center}
\includegraphics[width=.5\textwidth]{flow}
\end{center}
\caption{Flow Item example}\label{items:example:flow}
\end{figure}

Many items consist of several elements, for example flow item is a line
with
\begin{itemize}
\item stereotype
\item name
\item guard
\end{itemize}
These elements are aligned to a line of flow (see
figure~\ref{items:example:flow})
\begin{itemize}
\item stereotype is near one of the ends of line
\item name is under stereotype
\item guard is near the middle of line
\end{itemize}

Element of items are aligned according to attributes of
\rclass{ItemAlign} and \rclass{LineAlign} classes. Align information is
assigned to every item element, but position is calculated from item
perspective, for example
\begin{itemize}
\item align of an element to the (left, top) with margin $(10, 15, 15, 5)$
puts an element in (left, top) corner of item; it is located $10px$ from the top
and $5px$ from the left of item
\item align of an element as above but outside item puts an element $10px$
from top and $5px$ from the left of item; top and left margin values are
used for top vertical and left horizontal align
\end{itemize}
See figure~\ref{items:align} for more examples.

\begin{figure}
\begin{center}
\includegraphics[width=.95\textwidth]{align}
\end{center}
\caption{Align of Item Elements}\label{items:align}
\end{figure}

Align constants and classes are defined in \rmodule{gaphor::diagram::align} module.

\subsection{Stereotype and Name Align}

There are two important elements of items
\begin{itemize}
\item every item shown on diagram has stereotype (even item for Stereotype
    UML class, which stereotype is \rstereotype{stereotype})
\item many items has names
\end{itemize}

Therefore every item class has \rattr{s\_align} attribute, which defines
align of stereotypes.

Named items are aligned according to \rattr{n\_align} attribute. If
stereotype and name align values are the same, then name is put
centered under stereotype.

\warning{What about FeatureItem stereotype?}

\section{Profiles}

\subsection{Gaphor Stereotypes}

\begin{description}
\item[name] UML class of an item derives from NamedElement; name
        should be shown on diagram
\item[nostereotype] item does not allow to edit stereotypes
\end{description}

\subsection{Diacanvas Stereotypes}

\begin{description}
\item[editable] item is editable when it is possible to change name of an
    item by clicking twice on it
\end{description}


\section{Glossary}

\begin{description}
\item[item]
    item put on diagram by user, i.e. ClassItem, AssociationItem;
    conforms to UML notation of UML metamodel classes

\item[UML class]
    class defined in UML metamodel
\end{description}

\end{document}
